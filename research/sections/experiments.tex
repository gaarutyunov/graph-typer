To select the final model architecture, we test different models.
For our experiments and ablation analysis, we train and test the models using one sample repository.
We also limit the number of types in the vocabulary to one hundred to speed up training and use less resources.
To test the models, we calculate Top-n predictions similar to the previous work~\cite{mir_type4py_2021}.
Table~\ref{tab:ablation} depicts the results of the experiments and ablation.

\subsection{Validating the necessity of node and type identifiers that encode graph structure}\label{subsec:validating-the-necessity-of-node-and-type-identifiers-that-encode-graph-structure}

First of all, we remove the node and type identifiers introduced by Kim et. Al~\cite{kim_pure_2022} our ablation analysis demonstrates that indeed, the graph structure embeddings play a key role in model quality.
By removing them from the model, we are left with a simple Transformer that makes predictions only based on AST nodes and edges types without any information about graph structure.
Such a model outputs the worst results among all the experiments.

\subsection{Using the model without node type annotations}\label{subsec:using-the-model-without-node-type-annotations}

In addition, we try to remove the type annotations from the model completely.
This alteration turns our training into a masked NER task.
Surprisingly, our model performs well in such conditions.
This means that the selected graph representation of code contains a lot of necessary information to infer types.

\subsection{Increasing the number of parameters}\label{subsec:increasing-the-number-of-parameters}

As we can see, increasing the number of parameters also increases the predictive power of the model.
However, increasing the parameters indefinitely is not very practical and requires a lot of computational resources~\cite{arutyunov_big_2022}.
Therefore, we don't change the parameter number of the final model, so it remains compact.

\subsection{Testing different context length}\label{subsec:testing-different-context-length}

As for the context length, i.e., maximum number of nodes in graph (512 vs. 1024), our findings are aligned with the conclusions from previous work~\cite{arutyunov_big_2022}:
longer context increases the performance of the model.
However, the AST representation of source code is very bloated and even having a lot of nodes in the graph might not capture
enough useful information to make quality predictions.
In addition, increasing the context length drastically slows down the training process.
Thus, in future research, we will be working on finding a better and more compact graph representation of code.

\subsection{Testing different Transformer architectures}\label{subsec:testing-different-transformer-architectures-(encoder-only-vs-encoder-decoder)}

Recently, Masked Graph Autoencoders have been applied for the tasks of link prediction and node classification~\cite{tan2022mgae},
as well as feature reconstruction~\cite{zhang2022graph, hou2022graphmae}.
To validate the robustness of the Encoder-only Model, we also implement a Masked Autoencoder Model.
For this, we adapt the approach of Hou et. al~\cite{hou2022graphmae} for our model.
We introduce a learnable mask token and a decoder based on the encoder layers.
We reconstruct the type annotations by re-masking the target nodes before feeding them into the decoder.
However, we do not observe as good results as with a simple Encoder-only model.

\begin{table}
    \centering
    \caption{Expirement results of Top-n predictions for different model variants.}
    \label{tab:ablation}
    \begin{tabular}{llrrrr}
\toprule
 & \% Match & \multicolumn{2}{r}{Exact} & \multicolumn{2}{r}{Up to Parametric Type} \\
 & Context Length & 512 & 1024 & 512 & 1024 \\
Model & Top-n &  &  &  &  \\
\midrule
\multirow[t]{4}{*}{Base (51 mln)} & Top-1 & 30.88 & 31.12 & 36.55 & 35.83 \\
 & Top-3 & 40.33 & 42.79 & 50.37 & 56.30 \\
 & Top-5 & 42.82 & 45.30 & 56.01 & 60.12 \\
 & Top-10 & 48.03 & 48.72 & 64.98 & 64.59 \\
\cline{1-6}
\multirow[t]{4}{*}{Ablated (51 mln)} & Top-1 & 10.15 & 21.39 & 19.46 & 31.09 \\
 & Top-3 & 15.06 & 38.94 & 29.40 & 54.08 \\
 & Top-5 & 16.81 & 41.19 & 37.91 & 56.97 \\
 & Top-10 & 22.97 & 46.27 & 46.72 & 62.05 \\
\cline{1-6}
\multirow[t]{4}{*}{Deep (214 mln)} & Top-1 & 26.69 & 32.93 & 29.93 & 37.95 \\
 & Top-3 & 40.15 & 44.07 & 48.82 & 53.55 \\
 & Top-5 & 44.18 & 46.23 & 57.11 & 60.36 \\
 & Top-10 & 49.49 & 52.69 & 67.77 & 69.63 \\
\cline{1-6}
\multirow[t]{4}{*}{Big (331 mln)} & Top-1 & 30.90 & 32.77 & 33.59 & 38.29 \\
 & Top-3 & 42.75 & 43.36 & 49.62 & 51.46 \\
 & Top-5 & 47.28 & 45.21 & 58.25 & 55.92 \\
 & Top-10 & 52.72 & 50.25 & 68.53 & 66.20 \\
\cline{1-6}
\multirow[t]{4}{*}{Final (432 mln)} & Top-1 & 29.25 & 34.71 & 36.97 & 36.43 \\
 & Top-3 & 39.93 & 45.47 & 53.19 & 55.02 \\
 & Top-5 & 46.58 & 50.70 & 63.29 & 64.58 \\
 & Top-10 & 54.21 & 57.60 & 72.30 & 72.15 \\
\cline{1-6}
\bottomrule
\end{tabular}


\end{table}